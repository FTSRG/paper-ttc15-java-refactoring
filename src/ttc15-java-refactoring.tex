\documentclass[submission,copyright,creativecommons]{eptcs}
\providecommand{\event}{TTC 2015}

% TTC additions
\usepackage{graphicx}
\usepackage{subcaption}
\usepackage{xspace}
\usepackage{rotating,amsmath,amsfonts,amssymb}
\usepackage[T1]{fontenc}
\usepackage[utf8]{inputenc}
\usepackage{todonotes}
\usepackage{float}
\usepackage{xcolor}
\usepackage{listings}
\usepackage{framed}

% custom commands
\definecolor{negcolor}{RGB}{223,49,35}
\definecolor{newcolor}{RGB}{43,183,47}
\definecolor{delcolor}{RGB}{59,66,161}

\newcommand*\numcircledmod[1]{\raisebox{.5pt}{\textcircled{\raisebox{-.9pt} {\textsf{\footnotesize#1}}}}}
\newcommand{\union}{\cup}
\newcommand{\intersection}{\cap}
\newcommand{\parentheses}[1]{\left(#1\right)}
\newcommand{\op}[2]{\mathrm{#1}\parentheses{#2}}
\newcommand{\figref}[1]{\autoref{fig:#1}}
\newcommand{\lstref}[1]{\autoref{lst:#1}}
\newcommand{\naturaljoin}{\bowtie}
\newcommand{\antijoin}{\, \triangleright \,}
\newcommand{\eiq}{\mbox{\textsc{EMF-IncQuery}}\@\xspace}
\newcommand{\viatra}{\mbox{\textsc{Viatra}}\@\xspace}
\newcommand{\viatratwo}{\mbox{\textsc{Viatra2}}\@\xspace}
\newcommand{\evm}{\mbox{\textsc{Viatra-EVM}}\@\xspace}
\newcommand{\jdt}{\mbox{Eclipse JDT}\@\xspace}
\newcommand{\xtend}{\mbox{Xtend}\@\xspace}
\newcommand{\iqpl}{\mbox{\textsc{IncQuery}} Pattern Language\@\xspace}
\newcommand{\tb}{Train Benchmark\@\xspace}

\renewcommand{\sectionautorefname}{Section}
\renewcommand{\subsectionautorefname}{Section}
\renewcommand{\subsubsectionautorefname}{Section}

\newcommand{\ie}{i.e.\@\xspace}
\newcommand{\Ie}{I.e.\@\xspace}
\newcommand{\eg}{e.g.\@\xspace}
\newcommand{\Eg}{E.g.\@\xspace}
\newcommand{\etal}{et al.\@\xspace}
\newcommand{\etc}{etc.\@\xspace}

\newcommand{\ttcsf}[1]{\textsf{\small #1}}
\newcommand{\ttcfig}[3]{
\begin{figure}[htb]
	\centering
	\includegraphics[width=#3\textwidth]{figures/#1}
	\caption{#2.}
	\label{fig:#1}
\end{figure}}

\newcommand{\ttcpattern}[2]{
\begin{figure}[H] 
	\centering
	\includegraphics[scale=0.3]{figures/#1}
	\caption{#2.}
	\label{fig:#1}
\end{figure}}

\newcommand{\ttcfigscale}[3]{
\begin{figure}[htb]
	\centering
	\includegraphics[scale=#3]{figures/#1}
	\caption{#2.}
	\label{fig:#1}
\end{figure}}

\newcommand{\ttcsubfig}[2]{
	\begin{subfigure}[b]{0.49\textwidth}
		\centering
		\includegraphics[scale=0.21]{figures/#1}
		\caption{#2}
		\label{fig:#1}
	\end{subfigure}
}

\definecolor{lightgray}{RGB}{242,242,242}
\definecolor{keywordcolor}{RGB}{0,0,160}
\definecolor{commentcolor}{RGB}{0,128,64}
\lstset{
	numbers=left,
	numberstyle=\scriptsize\ttfamily,
	stepnumber=1,
	numbersep=5pt,
	%
	backgroundcolor=\color{lightgray},
	basicstyle=\scriptsize\ttfamily, % print whole listing small
	keywordstyle=\color{keywordcolor}\bfseries\ttfamily,
	commentstyle=\color{commentcolor}\ttfamily,
	stringstyle=\color{stringcolor}\ttfamily,
	identifierstyle=, % nothing happens
	stringstyle=\scriptsize,
	%
	showstringspaces=false, % no special string spaces
	aboveskip=3pt,
	belowskip=3pt,
	columns=flexible,
	keepspaces=true,
	breaklines=true,	
	frameround=tttt,
	captionpos=b,
	tabsize=2,
	frame=tb,
	framerule=0pt,
	framexleftmargin=0.25em,
}

\lstdefinelanguage{iqpl}
{
	morekeywords={@QueryBasedFeature,@Constraint,count,pattern,package,neg,find,import,true,false,or,check,job,action,state,severity,location,message,oclIsKindOf,self,exists,includes,invariant,class},
	sensitive=true,
	morecomment=[l]{//},
	morecomment=[s]{/*}{*/},
	morestring=[b]{"},
}

\lstdefinelanguage{xtend}{
	morekeywords={class,public,def,val,var,cached,case,default,extension,false,import,JAVA,WORKFLOWSLOT,let,new,null,private,create,switch,this,true,reexport,around,if,then,else,context,DEFAULT_NO_UPDATE_AND_DISAPPEAR,DEFAULT,APPEARED,DISAPPEARED,UPDATED,processor},
	keywordstyle=[2]{\textbf},
	morecomment=[l]{//},
	morecomment=[s]{/*}{*/},
	morestring=[b]",
	mathescape=true,
}

\newcommand{\listingiqpl}[2]
{
	\lstset{
		language=iqpl
	}
	\lstinputlisting[label=lst:#1, caption=#2.]{code/#1}
}

\newcommand{\listingxtend}[2]{
	\lstset{
		language=xtend,
		escapeinside={(*@}{@*)},
		literate=*{[guilleft]}{\guillemotleft{}}{1}{[guilright]}{\guillemotright{}}{1}
	}
	\lstinputlisting[label=lst:#1, caption=#2.]{code/#1}
}

\newcommand{\listingjava}[2]{
	\lstset{
		language=Java
	}
	\lstinputlisting[label=lst:#1, caption=#2.]{code/#1}
}

\usepackage{enumitem}

\newcommand{\pum}{\ttcsf{Pull Up Method}\@\xspace}
\newcommand{\cscz}{\ttcsf{Create Superclass}\@\xspace}

% metadata
\title{Java Refactoring Case: a \viatra Solution\thanks{This work was partially supported by the MONDO (EU ICT-611125) project.}}
\author{D\'{a}niel Stein \qquad G\'{a}bor Sz\'{a}rnyas \qquad Istv\'{a}n R\'{a}th
\institute{Budapest University of Technology and Economics\\
Department of Measurement and Information Systems\\
H-1117 Magyar tud\'{o}sok krt. 2, Budapest, Hungary}
\email{daniel.stein@inf.mit.bme.hu, \{szarnyas, rath\}@mit.bme.hu}
}
\def\titlerunning{Java Refactoring Case: a VIATRA Solution}
\def\authorrunning{D. Stein et al.}

\begin{document}
\maketitle

\begin{abstract}
This paper presents a solution for the Java Refactoring Case of the 2015 Transformation Tool Contest. The solution utilises \jdt for creating the program graph; while \eiq, \viatra and the \xtend programming language are used for defining and performing the model transformations.
\end{abstract}

\section{Introduction}

This paper describes a solution for the extended version of the TTC 2015 Java Refactoring Case. The source code of the solution is available as an open-source project.\footnote{\url{https://github.com/FTSRG/java-refactoring-ttc-viatra}} There is also a SHARE image available.\footnote{\url{http://is.ieis.tue.nl/staff/pvgorp/share/?page=ConfigureNewSession&vdi=ArchLinux64_java-refactoring-viatra.vdi}}

The use of automated model transformations is a key factor in modern model-driven system engineering. Model transformations allow the users to query, derive and manipulate large industrial models, including models based on existing systems, \eg source code models created with reverse engineering techniques. Since such transformations are frequently integrated to modeling environments, they need to feature both high performance and a concise programming interface to support software engineers. \eiq and \viatra aim to provide an expressive query language and a carefully designed API for defining model queries and transformations.

\section{Case Description}

Refactoring operations are often used in software engineering to improve the readability and maintainability of existing source code without altering the behaviour of the software. The goal of the Java Refactoring Case~\cite{ttc-refactoring-case} is to use model transformation tools to refactor Java source code. We decided to solve the extended version of the case. For this, the solution has to tackle the following challenges:

\begin{enumerate}[noitemsep]
\item Transforming the \emph{Java source code} to a \emph{program graph} (PG).
\item Performing the refactoring transformation on the program graph.
\item Synchronising the source code and the program graph.
\end{enumerate}

The source code is defined in a restricted sub-language of Java 1.4. The EMF metamodel of the PG is provided in the case description. The case considers two basic refactoring operations: \pum and \cscz.

The solution is tested in an automated test framework, ARTE (Automated Refactoring Test Environment). ARTE runs a set of test cases, each consisting of a source code projects and refactoring operations to check the correctness and the performance of the solution.

\section{Technologies}

Solving the case requires the integration of a model transformation tool and a Java source code parser. In this section, we introduce the technologies used in our solution.

\subsection{\eiq}

The objective of the \eiq~\cite{models2010, eiq-homepage} framework is to provide a declarative way to define queries over EMF models. \eiq extended the pattern language of \viatratwo with new features (including transitive closure, role navigation, match count) and tailored it to EMF models, resulting in \iqpl~\cite{iqpl}. While \eiq is developed with a focus on \emph{incremental query evaluation}, the latest version also provides a \emph{local search-based query evaluation} algorithm.

\subsection{\viatra}

The \viatra framework supports the development of model transformations with specific focus on event-driven, reactive transformations~\cite{viatra}. Building upon the incremental query support of \eiq, \viatra offers a language to define transformations and a reactive transformation engine to execute certain transformations upon changes in the underlying model. The \viatra project provides:

\begin{itemize}[noitemsep]
	\item An internal DSL over the \xtend~\cite{Xtend} language to specify both batch and event-driven, reactive transformations.
	\item A complex event-processing engine over EMF models to specify reactions upon detecting complex sequences of events.
	\item A rule-based design space exploration framework to explore design candidates as models satisfying multiple criteria.
	\item A model obfuscator to remove sensitive information from a confidential model, \eg for creating bug reports.
\end{itemize}

The current \viatra{} project is a full rewrite of the previous \viatratwo{} framework, now with full compatibility and support for EMF models. The history of the \viatra{} family is described in~\cite{viatra-history}.

\subsection{Java Development Tools}

The solution requires a technology to parse the Java code into a program graph model and serialize the modified graph model back to source code. While the case description mentions the JaMoPP~\cite{JaMoPP} and MoDisco~\cite{MoDisco} technologies, our solution builds on top of the Eclipse Java Development Tools (JDT)~\cite{jdt} used in the Eclipse Java IDE as we were already using JDT in other projects.

Compared to the MoDisco framework (which uses JDT internally), we found JDT to be simpler to deploy outside the Eclipse environment, \ie without defining an Eclipse workspace. Meanwhile, the JaMoPP project has almost completely been abandoned and therefore it is only capable of parsing Java 1.5 source files. While this would not pose a problem for this case, we think it is best to use an actively developed technology such as JDT which supports the latest (1.8) version of the Java language. As JDT is frequently used to parse large source code repositories, it is carefully optimised and supports lazy loading. Unlike JaMoPP and MoDisco, JDT does not produce an EMF model.

The JDT parser generates Abstract Syntax Trees (ASTs) from the provided source code files with an optional binding resolution mechanism. This way, the separate syntax trees can be connected to an Abstract Syntax Graph (ASG).



\section{Implementation}

The solution was developed partly in IntelliJ IDEA and partly in the Eclipse IDE. The projects are not tied to the any development environment and can be compiled with the Apache Maven~\cite{Maven} build automation tool. This offers a number of benefits, including easy portability and the possibility of continuous integration.

The code is written in Java~8 and Xtend~\cite{Xtend}. The queries and transformations were defined in \eiq and \viatra, respectively. For developing the Xtend code and editing the graph patterns, it is required to use the Eclipse IDE. For setting up the development environment, please refer to the readme file. 

\subsection{Workflow of the Transformation}

\ttcfig{pg}{Workflow of the transformation}{0.8}

\noindent\autoref{fig:pg} shows the workflow of the transformation. The workflow consists of five steps: the source code is parsed into an ASG~\ttccircled{1}; a PG is produced~\ttccircled{2}; based on the PG the possible transformations are calculated~\ttccircled{3}; if possible, these transformations are executed~\ttccircled{4}; finally, the results of the transformations are serialized~\ttccircled{5}. In this section, we discuss these steps in detail.

\subsection[Parsing the Source Code]{Parsing the Source Code\qquad\ttccircled{1}}
The framework receives the path to the directory containing the source files. JDT parses these files and returns each file, each compilation unit parsed as an AST. These ASTs are interconnected, which means that using JDT's binding resolution mechanism, the developer can navigate from one AST to another one.

\subsection[Producing the Program Graph]{Producing the Program Graph\qquad\ttccircled{2}}
Since JDT does not produce EMF models, it does not support complex queries and traversal operations as the ones provided by EMF and EMF-based query languages (e.g. Eclipse OCL or \eiq). To extract information and build the PG, our solution applies a visitor resulting a two-pass traversal on the ASG.

\begin{enumerate}[noitemsep]
\item For each object, the visitor method creates the corresponding object(s) in the PG. Since the order of these visits is non-deterministic, the framework uses maps to store the mapping from the objects in the ASG to the objects in the PG.

These maps provide trace information between the JDT model and the partially built PG. The visitor also collects the relations between JDT nodes and caches the unique identifiers of each connected node for every relation type. 

\item After every compilation unit was parsed, the previously populated caches are used to create the cross-references between the objects in the PG (\eg \ttctt{TMember.access}).
\end{enumerate}

\subsection[Extending the PG with the Trace Model]{Extending the PG with the Trace Model\qquad\ttccircled{2}\ttccircled{3}\ttccircled{4}}

\ttcfig{trace-mm}{Metamodel of the trace model}{0.95}

\noindent The main patterns for both refactoring operations contain a condition that \eiq does not support out of the box, \eg checking ``every child'' (a collection of classes) of a certain class. Passing collections as pattern parameter is only possible with a workaround. Also, the \iqpl does not support universal quantifiers. To overcome these limitations, we extended the program graph metamodel with a \emph{trace model} shown in~\autoref{fig:trace-mm}. The trace model defines traces for method signatures and class lists:

\begin{itemize}[noitemsep]
\item \ttcsf{MethodSignatureTrace.} Java methods are uniquely identifiable by their signature. The basic PG metamodel contains a \ttcsf{TMethodSignature} class, which only identifies itself with the name of the method (with a relation to the \ttcsf{TMethod} object) and the list of its parameter types.

To support querying \ttcsf{TMethodSignature} objects with \eiq, we created a trace reference for each of them identified by their (partial\footnote{The complete signature would also contain the defining type (class or interface) signature and the return type signature.}) method signature. For example, a method \ttcsf{method()} expecting a \ttcsf{String} and an \ttcsf{Integer} will have the \ttctt{.method(Ljava/lang/String;I)} trace signature.
\item \ttcsf{ClassListTrace.} To express the collection of classes, a \ttcsf{ClassListTrace} object will identify them with their signatures joined by the \ttctt{\#} character. For example, a list of the \ttcsf{ChildClass1} and \ttcsf{ChildClass2} classes in the \ttcsf{example04} package have the \ttctt{Lexample04/ChildClass1;\#Lexample04/ChildClass2;} list identifier.
\end{itemize}

After the PG is produced, it is extended with the trace model. The traces are based on \eiq patterns (\autoref{lst:TraceQueries.eiq}) and generated with a \viatra transformation (\autoref{lst:TraceTransformation.xtend}).

The \emph{universal quantifier} is implemented as a double negation of the existential quantifier using the well-known identity $(\forall x) P(x) \Leftrightarrow \neg (\exists x) \neg P(x)$.

\subsection[Refactoring]{Refactoring\qquad\ttccircled{3}\ttccircled{4}}

The refactoring operations are implemented as model transformations on the JDT ASG and the PG. Each model transformation is defined in \viatra: the LHS is defined with an \eiq pattern and the RHS is defined with an imperative Xtend code. As \viatra does not support bidirectional transformations, for each transformation on the PG, we also execute the appropriate actions on the ASG to keep the two graphs in sync.

\subsubsection{Pull Up Method}
After creating the method signature traces, the following preconditions must be met before pulling up a method:

\begin{itemize}[noitemsep]
	 \item every child class has a method with the given signature,
	 \item the parent class does not have a method with this signature,
	 \item the transformation will not create an unsatisfiable method or field access.
\end{itemize}

The main pattern---deciding whether the \pum can be executed---collects every parent class--method signature pair satisfying the preconditions. The LHS is defined with six patterns in total. The execution is controlled by parameterising the main pattern (\ttcsf{possiblePUM}) listed in \autoref{lst:PUMQueries.eiq}. The RHS is defined in \autoref{lst:PUMTransformation.xtend}, using one utility pattern.

\subsubsection{Create Superclass}
To create a new superclass, the parent class and the list of selected classes (connected to a class list trace) have to be passed to the pattern. The transformation can be executed if the following preconditions are satisfied:

\begin{itemize}[noitemsep]
	\item the target parent class does not exist,
	\item every selected child class has the same parent.
\end{itemize}

The LHS is defined in~\autoref{lst:CSCQueries.eiq} with the \ttcsf{possibleCSC} pattern using five other patterns. The RHS is defined in \autoref{lst:CSCTransformation.xtend}, also using a utility pattern.

\subsection[Transforming the ASG to Source Code]{Transforming the ASG to Source Code\qquad\ttccircled{5}}

The changes in the ASG made by the transformations are propagated to the source code. JDT is capable of incrementally maintaining each source code file (compilation unit) based on the changes in its AST.\footnote{The \textsf{CompilationUnit.rewrite()} method returns a set of text manipulation operations in the form of a \textsf{TextEdit} object.}

\section{Evaluation}

The benchmarks were conducted on SHARE, running a 64-bit Arch Linux virtual machine. The machine utilized a single core of a 2.00~GHz Xeon E5-2650 CPU and 1~GB of RAM. We used OpenJDK~8 to run ARTE framework and the solution.

\subsection{Benchmark Results}

We executed the tests and used the log files to determine the execution times. The execution times of the test cases are listed in \autoref{tab:execution-times}. 

\begin{table}[!htb]
\centering
\footnotesize
\begin{tabular}{| l | r |}
\hline
\bf test case & \bf time [s]\\\hline\hline
\sf pub\_pum1\_1\_paper1 & 0.463 \\\hline
\sf pub\_pum1\_2 & 0.013 \\\hline
\sf pub\_pum2\_1 & 0.333 \\\hline
\sf pub\_pum3\_1 & 0.094 \\\hline
\sf pub\_csc1\_1 & 0.189 \\\hline
\sf pub\_csc1\_2 & 0.093 \\\hline
\sf hidden\_pum1\_1 & 0.063 \\\hline
\sf hidden\_pum1\_2 & 0.013 \\\hline
\sf hidden\_pum2\_1 & 0.114 \\\hline
\sf hidden\_pum2\_2 & 0.082 \\\hline
\sf hidden\_csc1\_1 & 0.081 \\\hline
\sf hidden\_csc1\_2 & 0.007 \\\hline
\sf hidden\_csc2\_1 & 0.058 \\\hline
\sf hidden\_csc3\_1a & 0.189 \\\hline
\sf hidden\_csc3\_1 & 0.179 \\\hline

\end{tabular}\caption{Execution times for the test cases.}
\label{tab:execution-times}
\end{table}

\subsection{Analysis of the Solution}

The results show that all public and hidden test cases have been executed successfully. Hence, we consider the solution as \emph{complete} and \emph{correct}. As the test cases only contained small examples, we cannot draw conclusions on the performance of the solution. However, it is worth noting that all test cases executed in less than half a second.

The implementation of the solution required quite a lot of code. The patterns were formulated in about 150 lines of \iqpl code. The transformations required 400 lines of Xtend code, while implementing the interface required by ARTE and the visitor for the transformation required more than 800 lines of Java code.

\section{Summary}

This paper presented a solution for the Java Refactoring case of the 2015 Transformation Tool Contest. The solution addresses both challenges (bidirectional synchronisation and program transformation) and both refactoring operations (pull up method, create superclass) defined in the case. The framework is flexible enough to allow the user to define new refactoring operations, \eg \ttcsf{Extract Class} or \ttcsf{Pull Up Field}.

\paragraph{Acknowledgements.} The authors would like to thank Ábel Hegedüs, Oszkár Semeráth and Zoltán Ujhelyi for providing valuable insights into \eiq and \viatra.

\bibliographystyle{eptcs}
\bibliography{ttc}

\clearpage

\appendix
\section{Appendix}

\subsection{Patterns}

\listingiqpl{TraceQueries.eiq}{Patterns for generating the trace model}

\listingiqpl{PUMQueries.eiq}{Patterns for the \pum refactoring}

\listingiqpl{CSCQueries.eiq}{Patterns for the \cscz refactoring}

\subsection{Transformations}

\listingxtend{TraceTransformation.xtend}{Transformation for generating the trace model}

\listingxtend{PUMTransformation.xtend}{\pum transformation}

\listingxtend{CSCTransformation.xtend}{\cscz transformation}

\end{document}
