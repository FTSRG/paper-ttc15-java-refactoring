% TTC additions
\usepackage{graphicx}
\usepackage{subcaption}
\usepackage{xspace}
\usepackage{rotating,amsmath,amsfonts,amssymb}
\usepackage[T1]{fontenc}
\usepackage[utf8]{inputenc}
\usepackage{todonotes}
\usepackage{float}
\usepackage{xcolor}
\usepackage{listings}
\usepackage{framed}

% custom commands
\definecolor{negcolor}{RGB}{223,49,35}
\definecolor{newcolor}{RGB}{43,183,47}
\definecolor{delcolor}{RGB}{59,66,161}

\newcommand*\numcircledmod[1]{\raisebox{.5pt}{\textcircled{\raisebox{-.9pt} {\textsf{\footnotesize#1}}}}}
\newcommand{\union}{\cup}
\newcommand{\intersection}{\cap}
\newcommand{\parentheses}[1]{\left(#1\right)}
\newcommand{\op}[2]{\mathrm{#1}\parentheses{#2}}
\newcommand{\figref}[1]{\autoref{fig:#1}}
\newcommand{\lstref}[1]{\autoref{lst:#1}}
\newcommand{\naturaljoin}{\bowtie}
\newcommand{\antijoin}{\, \triangleright \,}
\newcommand{\eiq}{\mbox{\textsc{EMF-IncQuery}}\@\xspace}
\newcommand{\viatra}{\mbox{\textsc{Viatra}}\@\xspace}
\newcommand{\viatratwo}{\mbox{\textsc{Viatra2}}\@\xspace}
\newcommand{\evm}{\mbox{\textsc{Viatra-EVM}}\@\xspace}
\newcommand{\jdt}{\mbox{Eclipse JDT}\@\xspace}
\newcommand{\xtend}{\mbox{Xtend}\@\xspace}
\newcommand{\iqpl}{\mbox{\textsc{IncQuery}} Pattern Language\@\xspace}
\newcommand{\tb}{Train Benchmark\@\xspace}

\renewcommand{\sectionautorefname}{Section}
\renewcommand{\subsectionautorefname}{Section}
\renewcommand{\subsubsectionautorefname}{Section}

\newcommand{\ie}{i.e.\@\xspace}
\newcommand{\Ie}{I.e.\@\xspace}
\newcommand{\eg}{e.g.\@\xspace}
\newcommand{\Eg}{E.g.\@\xspace}
\newcommand{\etal}{et al.\@\xspace}
\newcommand{\etc}{etc.\@\xspace}

\newcommand{\ttcsf}[1]{\textsf{\small #1}}
\newcommand{\ttcfig}[3]{
\begin{figure}[htb]
	\centering
	\includegraphics[width=#3\textwidth]{figures/#1}
	\caption{#2.}
	\label{fig:#1}
\end{figure}}

\newcommand{\ttcpattern}[2]{
\begin{figure}[H] 
	\centering
	\includegraphics[scale=0.3]{figures/#1}
	\caption{#2.}
	\label{fig:#1}
\end{figure}}

\newcommand{\ttcfigscale}[3]{
\begin{figure}[htb]
	\centering
	\includegraphics[scale=#3]{figures/#1}
	\caption{#2.}
	\label{fig:#1}
\end{figure}}

\newcommand{\ttcsubfig}[2]{
	\begin{subfigure}[b]{0.49\textwidth}
		\centering
		\includegraphics[scale=0.21]{figures/#1}
		\caption{#2}
		\label{fig:#1}
	\end{subfigure}
}

\definecolor{lightgray}{RGB}{242,242,242}
\definecolor{keywordcolor}{RGB}{0,0,160}
\definecolor{commentcolor}{RGB}{0,128,64}
\lstset{
	numbers=left,
	numberstyle=\scriptsize\ttfamily,
	stepnumber=1,
	numbersep=5pt,
	%
	backgroundcolor=\color{lightgray},
	basicstyle=\scriptsize\ttfamily, % print whole listing small
	keywordstyle=\color{keywordcolor}\bfseries\ttfamily,
	commentstyle=\color{commentcolor}\ttfamily,
	stringstyle=\color{stringcolor}\ttfamily,
	identifierstyle=, % nothing happens
	stringstyle=\scriptsize,
	%
	showstringspaces=false, % no special string spaces
	aboveskip=3pt,
	belowskip=3pt,
	columns=flexible,
	keepspaces=true,
	breaklines=true,	
	frameround=tttt,
	captionpos=b,
	tabsize=2,
	frame=tb,
	framerule=0pt,
	framexleftmargin=0.25em,
}

\lstdefinelanguage{iqpl}
{
	morekeywords={@QueryBasedFeature,@Constraint,count,pattern,package,neg,find,import,true,false,or,check,job,action,state,severity,location,message,oclIsKindOf,self,exists,includes,invariant,class},
	sensitive=true,
	morecomment=[l]{//},
	morecomment=[s]{/*}{*/},
	morestring=[b]{"},
}

\lstdefinelanguage{xtend}{
	morekeywords={class,public,def,val,var,cached,case,default,extension,false,import,JAVA,WORKFLOWSLOT,let,new,null,private,create,switch,this,true,reexport,around,if,then,else,context,DEFAULT_NO_UPDATE_AND_DISAPPEAR,DEFAULT,APPEARED,DISAPPEARED,UPDATED,processor},
	keywordstyle=[2]{\textbf},
	morecomment=[l]{//},
	morecomment=[s]{/*}{*/},
	morestring=[b]",
	mathescape=true,
}

\newcommand{\listingiqpl}[2]
{
	\lstset{
		language=iqpl
	}
	\lstinputlisting[label=lst:#1, caption=#2.]{code/#1}
}

\newcommand{\listingxtend}[2]{
	\lstset{
		language=xtend,
		escapeinside={(*@}{@*)},
		literate=*{[guilleft]}{\guillemotleft{}}{1}{[guilright]}{\guillemotright{}}{1}
	}
	\lstinputlisting[label=lst:#1, caption=#2.]{code/#1}
}

\newcommand{\listingjava}[2]{
	\lstset{
		language=Java
	}
	\lstinputlisting[label=lst:#1, caption=#2.]{code/#1}
}
